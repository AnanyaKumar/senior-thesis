% LaTeX Article Template - using defaults
\documentclass[12pt]{article}
\usepackage{amsmath}
\usepackage{amssymb}
\usepackage{amsfonts}
\usepackage{amsthm}
\usepackage{mathrsfs}
\pdfpagewidth 8.5in
\pdfpageheight 11in

\usepackage[final]{showkeys}


% Set left margin - The default is 1 inch, so the following 
% command sets a 1.25-inch left margin.
\setlength{\oddsidemargin}{0.25in}

% Set width of the text - What is left will be the right margin.
% In this case, right margin is 8.5in - 1.25in - 6in = 1.25in.
\setlength{\textwidth}{6in}

% Set top margin - The default is 1 inch, so the following 
% command sets a 0.75-inch top margin.
\setlength{\topmargin}{-0.25in}

% Set height of the text - What is left will be the bottom margin.
% In this case, bottom margin is 11in - 0.75in - 9.5in = 0.75in
\setlength{\textheight}{8in}


% Set the beginning of a LaTeX document
\begin{document}

\title{Senior Thesis Proposal}         % Enter your title between curly braces
\author{Ananya Kumar (ananyak)}        % Enter your name between curly braces
\date{\today}          % Enter your date or \today between curly braces
\maketitle

I would like to work with Prof. Avrim Blum on algorithms for coresets and on the applications of coresets to machine learning.

\section{Research Area}

Given a set of points $P$, a coreset $Q \subseteq P$ approximates $P$ with respect to some \emph{extent measure}. The notion of a coreset depends on the definition of the extent measure. A simple type of coreset is an approximate convex hull, where every point in P is within distance $\epsilon$ from the convex hull of the coreset $Q$.

Coresets have numerous applications in computational geometry and machine learning. Blum et al explain how approximate convex hulls can be used for sparse non-negative matrix factorization, a useful technique in unsupervised learning, and give offline algorithms for finding approximate convex hulls \cite{blum-peled}. Agarwal et al give a survey of computational geometry algorithms that can be approximately solved using coresets \cite{survey}.

\section{Research Questions}

In my senior thesis I intend to work on the following questions:

\begin{enumerate}
\item \textbf{[Streaming Algorithm]} Devise a streaming algorithm for approximate convex hulls (or other types of coresets) that is asymptotically optimal in terms of the number of points stored. To simplify the problem, we could assume that the point set $P$ is generated in a certain specified way.

\item \textbf{[Topic Recovery]} Assuming that the points in $P$ are generated by taking convex combinations of $k$ points in a point set $T$, recover the set $T$.

\item \textbf{[Parallel Reconstruction]} Given a point $p$ in the convex hull of $Q$, we can efficiently find a sparse convex combination of points in $Q$ that sum to $p$. However, existing algorithms that do this are inherently sequential - can we find an efficient parallel algorithm?

\item \textbf{[Supervised Learning Applications]} Coresets learn the structure of a data set - can we use this in meaningful ways in supervised learning tasks?

\end{enumerate}

\section{Research Plan}

In a previous independent study I came up with basic results for the topic recovery problem, and examined a supervised learning application of coresets. I will build on this work and expect to have algorithms with proofs for some of the above research questions.

I will begin by reading \cite{survey} for a general survey of coresets, \cite{Hershberger2008}, \cite{adaptivesampling}, \cite{Clarkson1993} for previous work on streaming algorithms for coresets, and \cite{cvms}, \cite{icml2015_bachem15}, \cite{Feldman} for applications of coresets to machine learning. By the end of October 2016 I will conjecture algorithms for some of the research questions, and by the end of December 2016 I will have proofs for some of these conjectures.

% We recommend abbrvnat bibliography style.

\bibliographystyle{alpha}

% The bibliography should be embedded for final submission.

\bibliography{references}


\end{document}